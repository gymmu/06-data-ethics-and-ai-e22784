\documentclass{article}

\usepackage[ngerman]{babel}
\usepackage[utf8]{inputenc}
\usepackage[T1]{fontenc}
\usepackage{hyperref}
\usepackage{csquotes}

\usepackage[
    backend=biber,
    style=apa,
    sortlocale=de_DE,
    natbib=true,
    url=false,
    doi=false,
    sortcites=true,
    sorting=nyt,
    isbn=false,
    hyperref=true,
    backref=false,
    giveninits=false,
    eprint=false]{biblatex}
\addbibresource{../references/bibliography.bib}

\title{Notizen zum Projekt Data Ethics}
\author{Jaël Theiler}
\date{\today}

\begin{document}
\maketitle

\abstract{
    Cookies sind Textdateien

    \begin{itemize}
        \renewcommand{\labelitemi}{$\rightarrow$}
\item bleiben auf dem Gerät bis man die Webseite das nächste mal besucht --> dann werden sie wieder auf den Server übertragen/verfallsdatum (schliessen von website - mehrere Jahre)

\item Online- Marketing
\item Einwilligung der Benutzern
\item nicht notwendige blockiert ->  EU-Datenschutzgrundverordnung/Senibilisierung/
\item nicht hohe überlebenschancen

\item Warenkörbe
\item Passwörter
\item Voreinstellungen/Benutzereinstellungen
\item User-Verhalten

    \end{itemize}

\textbf{Welche Datenschutzprobleme entstehen durch die Nutzung von Cookies?}

\textbf{Welche Verantwortung tragen Unternehmen, die Cookies für ihr Online-Marketing benutzen in Bezug auf die Privatsphäre der Nutzer?}
}

\tableofcontents

\section{Einleitung}


\printbibliography

\end{document}
