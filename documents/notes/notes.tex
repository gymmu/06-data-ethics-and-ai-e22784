\documentclass{article}

\usepackage[ngerman]{babel}
\usepackage[utf8]{inputenc}
\usepackage[T1]{fontenc}
\usepackage{hyperref}
\usepackage{csquotes}

\usepackage[
    backend=biber,
    style=apa,
    sortlocale=de_DE,
    natbib=true,
    url=false,
    doi=false,
    sortcites=true,
    sorting=nyt,
    isbn=false,
    hyperref=true,
    backref=false,
    giveninits=false,
    eprint=false]{biblatex}
\addbibresource{../references/bibliography.bib}

\title{Notizen zum Projekt Data Ethics}
\author{Jaël Theiler}
\date{\today}

\begin{document}
\maketitle

\abstract{}

\tableofcontents

\section{Einleitung}
Künstliche Intelligenz (KI) ermöglicht es Computern, Aufgaben zu erledigen, die normalerweise menschliche Intelligenz erfordern. Zum Beispiel Lernen, Problemlösen und Sprachverstehen. Im Alltag begegnen wir der KI zum Beispiel in Sprachassistenten wie Siri und Alexa, Empfehlungssysteme auf Plattformen wie Netflix und Amazon sowie bei den selbstfahrende Autos.
Um die KI zu trainieren wird z.B. das maschinelle Lernen (ML) angewendet, bei dem der Computer aus großen Datenmengen Muster erkennt und analysiert.Neuronale Netze, die dem menschlichen Gehirn nachempfinden, spielen dabei eine wichtige Rolle.Das Neuronale Netz ist ein Modell, welches mit Prozessen nachahmen soll wie biologische Neuronen zusammenarbeiten um möglichst Entscheidungen so zu treffen wie das menschliche Gehirn es tun würde. So versucht die KI möglichst verständliche Schlussfolgerungen zuziehen und seine Antworten zu optimieren.

\section{Notizen}
Cookies sind Textdateien

    \begin{itemize}
        \renewcommand{\labelitemi}{$\rightarrow$}
\item bleiben auf dem Gerät bis man die Webseite das nächste mal besucht --> dann werden sie wieder auf den Server übertragen/verfallsdatum (schliessen von website - mehrere Jahre)

\item Online- Marketing
\item Einwilligung der Benutzern
\item nicht notwendige blockiert 
\item EU-Datenschutzgrundverordnung/Senibilisierung/
\item nicht hohe überlebenschancen

\item Warenkörbe
\item Passwörter
\item Voreinstellungen/Benutzereinstellungen
\item User-Verhalten

    \end{itemize}

\textbf{Welche Datenschutzprobleme entstehen durch die Nutzung von Cookies?}

\textbf{Welche Verantwortung tragen Unternehmen, die Cookies für ihr Online-Marketing benutzen in Bezug auf die Privatsphäre der Nutzer?}

\printbibliography
 

\end{document}
