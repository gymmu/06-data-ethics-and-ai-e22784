\documentclass{article}

\usepackage[ngerman]{babel}
\usepackage[utf8]{inputenc}
\usepackage[T1]{fontenc}
\usepackage{hyperref}
\usepackage{csquotes}

\usepackage[
    backend=biber,
    style=apa,
    sortlocale=de_DE,
    natbib=true,
    url=false,
    doi=false,
    sortcites=true,
    sorting=nyt,
    isbn=false,
    hyperref=true,
    backref=false,
    giveninits=false,
    eprint=false]{biblatex}
\addbibresource{../references/bibliography.bib}

\title{Notizen zum Projekt Data Ethics}
\author{Jaël Theiler}
\date{\today}

\begin{document}
\maketitle

\abstract{}

\tableofcontents

\section{Einleitung}
Künstliche Intelligenz (KI) ermöglicht es Computern, Aufgaben zu erledigen, die normalerweise menschliche Intelligenz erfordern. Zum Beispiel Lernen, Problemlösen und Sprachverstehen. Im Alltag begegnen wir der KI zum Beispiel in Sprachassistenten wie Siri und Alexa, Empfehlungssysteme auf Plattformen wie Netflix und Amazon sowie bei den selbstfahrende Autos.
Um die KI zu trainieren wird z.B. das maschinelle Lernen (ML) angewendet, bei dem der Computer aus grossen Datenmengen Muster erkennt und analysiert. Neuronale Netze, die dem menschlichen Gehirn nachempfinden, spielen dabei eine wichtige Rolle. Das Neuronale Netz ist ein Modell, welches mit Prozessen nachahmen soll wie biologische Neuronen zusammenarbeiten um möglichst Entscheidungen so zu treffen wie das menschliche Gehirn es tun würde. So versucht die KI möglichst verständliche Schlussfolgerungen zuziehen und seine Antworten zu optimieren.

\section{Was sind Cookies?}
Cookies sind einfach gesagt Textdateien, die bei einem Besuch einer Webseite von einem Server auf das Gerät des Users übertragen werden. Diese Dateien enthalten Informationen, die es der Webseite ermöglichen, den Nutzer bei zukünftigen Besuchen wiederzuerkennen und bestimmte Einstellungen oder Präferenzen beizubehalten. Diese Textdateien bleiben auch auf diesem Gerät bis sie entweder gelöscht werden, ihr Verfallsdatum erreichen oder man die Website wieder besucht, denn dann werden sie wieder auf den Server übertragen. Die Verfallsdaten, können  von der Schliessung der Website bis mehrere Jahre nach dem Besuch dauern.Die Dauer hängt von dem Zweck der Cookies ab. Cookies spielen eine wichtige Rolle im modernen Web, da es Webseiten ermöglichen, effizienter zu arbeiten. Sie werden unter anderem für Funktionen wie die Benutzeranmeldung, die Personalisierung von Inhalten und die Analyse des Nutzerverhaltens verwendet. Viele Unternehmen nutzen Cookies wegen deisen und noch ein paar anderen Gründen.

\section{Arten von Cookies}
Zusätzlich zu den standart Funktionen von Cookies gibt es noch verschiedene Arten davon, welche andere Aufgaben und Zwecke erfüllen. Dafür kann man sie zuerst in zwei Untergruppen teilen; den technisch notwendigen Cookies und den technisch nicht notwendigen Cookies. 
Technisch notwendige Cookies sind zum Beispiel Session Cookies. Sie speichern deine User Einstellungen wie auch Spracheinstellungen. Paypal ist beispielsweise ein Nutzer von Session Cookies. Wenn du dich bei deinem PayPal-Konto anmeldest, setzt PayPal ein Cookie auf dein Gerät, um dich während deiner Browsersitzung angemeldet zu halten. Dadurch musst du dich nicht bei jeder Transaktion erneut anmelden.
Diese Cookies werden auch verwendet, um die Sicherheit deines PayPal-Kontos zu gewährleisten. Sie können auch dazu beitragen, betrügerische Aktivitäten zu erkennen, indem sie Informationen über deine früheren Transaktionen und Anmeldeaktivitäten speichern und analysieren. Diese Art von Cookies dienen alleine zur Funktionsfähigkeit und Sicherheit einer Webseite.
Die technisch nicht notwendigen Cookies werden eher aus anderen Gründen genutzt. Sie dienen oft dazu, zusätzliche Funktionen zu ermöglichen oder das Nutzererlebnis zu verbessern. Jedoch sind sie nicht essenziell für Kernfunktionen einer Webseite. Beispiele für diese Cookies sind Marketing-,Analyse-, Statistik-, Personalisierungs- und Social Media- Cookies. Die Marketing Cookies verfolgen das Verhalten der Nutzer um personalisierte Werbung anzuzeigen. Andere sind dazu da den Betreibern dieser Webseiten zu helfen die Popularität ihrer Inhalte zu verstehen und auch die Anzahl der Nutzer zu messen.


\section{Notizen}
Cookies sind Textdateien

    \begin{itemize}
        \renewcommand{\labelitemi}{$\rightarrow$}
\item bleiben auf dem Gerät bis man die Webseite das nächste mal besucht --> dann werden sie wieder auf den Server übertragen/verfallsdatum (schliessen von website - mehrere Jahre)

\item Online- Marketing
\item Einwilligung der Benutzern
\item nicht notwendige blockiert 
\item EU-Datenschutzgrundverordnung/Senibilisierung/
\item nicht hohe überlebenschancen

\item Warenkörbe
\item Passwörter
\item Voreinstellungen/Benutzereinstellungen
\item User-Verhalten

    \end{itemize}

\textbf{Welche Datenschutzprobleme entstehen durch die Nutzung von Cookies?}

\textbf{Welche Verantwortung tragen Unternehmen, die Cookies für ihr Online-Marketing benutzen in Bezug auf die Privatsphäre der Nutzer?}



\printbibliography
 

\end{document}
